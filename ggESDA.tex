\documentclass[article]{jss}

%% -- LaTeX packages and custom commands ---------------------------------------

%% recommended packages
\usepackage{thumbpdf,lmodern}

%% another package (only for this demo article)
\usepackage{framed}

%% new custom commands
%\newcommand{\class}[1]{`\code{#1}'}
%\newcommand{\fct}[1]{\code{#1()}}

%% For Sweave-based articles about R packages:
%% need no \usepackage{Sweave}
%\SweaveOpts{engine=R, eps=FALSE, keep.source = TRUE}


%% -- Article metainformation (author, title, ...) -----------------------------

%% - \author{} with primary affiliation
%% - \Plainauthor{} without affiliations
%% - Separate authors by \And or \AND (in \author) or by comma (in \Plainauthor).
%% - \AND starts a new line, \And does not.
\author{Bo-Syue Jiang\\National Taipei University}
%\Plainauthor{Bo-Syue Jiang}

%% - \title{} in title case
%% - \Plaintitle{} without LaTeX markup (if any)
%% - \Shorttitle{} with LaTeX markup (if any), used as running title
\title{ggESDA: An \proglang{R} Package for Exploratory Symbolic Data Analysis using \pkg{ggplot2}}
%Plaintitle{A Short Demo Article: Regression Models for Count Data in R}
%\Shorttitle{A Short Demo Article in \proglang{R}}

%% - \Abstract{} almost as usual
\Abstract{
  This paper presents the \pkg{ggESDA} package,which we developed for exploratory symbolic data analysis in \proglang{R}.Based on \pkg{ggplot2} \cite{Wickham:2009}, the \pkg{ggESDA} package which is familiar programming structure with its parent provides a wide variety of graphical techniques such as histogram, 3D-scatterplot and radar.In addition, a general ,customized and user-friendly transformation function \code{classic2sym()} is implemented 
}

%% - \Keywords{} with LaTeX markup, at least one required
%% - \Plainkeywords{} without LaTeX markup (if necessary)
%% - Should be comma-separated and in sentence case.
\Keywords{SDA, EDA, symbolic data analysis, exploratory data analysis, ggplot2 extensions, interval-valued data, \proglang{R}}
\Plainkeywords{JSS, style guide, comma-separated, not capitalized, R}

%% - \Address{} of at least one author
%% - May contain multiple affiliations for each author
%%   (in extra lines, separated by \emph{and}\\).
%% - May contain multiple authors for the same affiliation
%%   (in the same first line, separated by comma).


\begin{document}
%\SweaveOpts{concordance=TRUE} modify this


%% -- Introduction -------------------------------------------------------------

%% - In principle "as usual".
%% - But should typically have some discussion of both _software_ and _methods_.
%% - Use \proglang{}, \pkg{}, and \code{} markup throughout the manuscript.
%% - If such markup is in (sub)section titles, a plain text version has to be
%%   added as well.
%% - All software mentioned should be properly \cite-d.
%% - All abbreviations should be introduced.
%% - Unless the expansions of abbreviations are proper names (like "Journal
%%   of Statistical Software" above) they should be in sentence case (like
%%   "generalized linear models" below).


%% -- Manuscript ---------------------------------------------------------------

%% - In principle "as usual" again.
%% - When using equations (e.g., {equation}, {eqnarray}, {align}, etc.
%%   avoid empty lines before and after the equation (which would signal a new
%%   paragraph.
%% - When describing longer chunks of code that are _not_ meant for execution
%%   (e.g., a function synopsis or list of arguments), the environment {Code}
%%   is recommended. Alternatively, a plain {verbatim} can also be used.
%%   (For executed code see the next section.)




%% -- Illustrations ------------------------------------------------------------

%% - Virtually all JSS manuscripts list source code along with the generated
%%   output. The style files provide dedicated environments for this.
%% - In R, the environments {Sinput} and {Soutput} - as produced by Sweave() or
%%   or knitr using the render_sweave() hook - are used (without the need to
%%   load Sweave.sty).
%% - Equivalently, {CodeInput} and {CodeOutput} can be used.
%% - The code input should use "the usual" command prompt in the respective
%%   software system.
%% - For R code, the prompt "R> " should be used with "+  " as the
%%   continuation prompt.
%% - Comments within the code chunks should be avoided - these should be made
%%   within the regular LaTeX text.


%
\begin{Schunk}
\begin{Sinput}
R> #R code in here
\end{Sinput}
\end{Schunk}
%

\includegraphics{ggESDA-visualization}




%





%% -- Summary/conclusions/discussion -------------------------------------------




%% -- Optional special unnumbered sections -------------------------------------



%% -- Bibliography -------------------------------------------------------------
%% - References need to be provided in a .bib BibTeX database.
%% - All references should be made with \cite, \citet, \citep, \citealp etc.
%%   (and never hard-coded). See the FAQ for details.
%% - JSS-specific markup (\proglang, \pkg, \code) should be used in the .bib.
%% - Titles in the .bib should be in title case.
%% - DOIs should be included where available.

\bibliography{refs}


%% -- Appendix (if any) --------------------------------------------------------
%% - After the bibliography with page break.
%% - With proper section titles and _not_ just "Appendix".

\newpage

\begin{appendix}

(wait for edit)

\end{appendix}

%% -----------------------------------------------------------------------------

%test
\end{document}
