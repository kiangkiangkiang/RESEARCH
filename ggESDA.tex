\documentclass[article]{jss}

%% -- LaTeX packages and custom commands ---------------------------------------

%% recommended packages
\usepackage{thumbpdf,lmodern}

%% another package (only for this demo article)
\usepackage{framed}

%% new custom commands
\newcommand{\class}[1]{`\code{#1}'}
\newcommand{\fct}[1]{\code{#1()}}

%% For Sweave-based articles about R packages:
%% need no \usepackage{Sweave}



%% -- Article metainformation (author, title, ...) -----------------------------

%% - \author{} with primary affiliation
%% - \Plainauthor{} without affiliations
%% - Separate authors by \And or \AND (in \author) or by comma (in \Plainauthor).
%% - \AND starts a new line, \And does not.
\author{Bo-Syue Jiang\\University Taipei
   \And Han-Ming Wu\\wait for edit}
\Plainauthor{Bo-Syue Jiang, Han-Ming Wu}

%% - \title{} in title case
%% - \Plaintitle{} without LaTeX markup (if any)
%% - \Shorttitle{} with LaTeX markup (if any), used as running title
\title{ggESDA: An \proglang{R} Package for exploratory symbolic data analysis using \pkg{ggplot2}}
\Plaintitle{A Short Demo Article: Regression Models for Count Data in R}
\Shorttitle{A Short Demo Article in \proglang{R}}

%% - \Abstract{} almost as usual
\Abstract{
  This paper presents the \pkg{ggESDA} package,which we developed for exploratory symbolic data analysis in \proglang{R}.Based on \pkg{ggplot2} \cite{Wickham:2009}
}

%% - \Keywords{} with LaTeX markup, at least one required
%% - \Plainkeywords{} without LaTeX markup (if necessary)
%% - Should be comma-separated and in sentence case.
\Keywords{SDA, EDA, symbolic data analysis, exploratory data analysis, ggplot2 extensions, interval-valued data, \proglang{R}}
\Plainkeywords{JSS, style guide, comma-separated, not capitalized, R}

%% - \Address{} of at least one author
%% - May contain multiple affiliations for each author
%%   (in extra lines, separated by \emph{and}\\).
%% - May contain multiple authors for the same affiliation
%%   (in the same first line, separated by comma).
\Address{
  myaddress
}

\begin{document}
%\SweaveOpts{concordance=TRUE} modify this


%% -- Introduction -------------------------------------------------------------

%% - In principle "as usual".
%% - But should typically have some discussion of both _software_ and _methods_.
%% - Use \proglang{}, \pkg{}, and \code{} markup throughout the manuscript.
%% - If such markup is in (sub)section titles, a plain text version has to be
%%   added as well.
%% - All software mentioned should be properly \cite-d.
%% - All abbreviations should be introduced.
%% - Unless the expansions of abbreviations are proper names (like "Journal
%%   of Statistical Software" above) they should be in sentence case (like
%%   "generalized linear models" below).

\section[Introduction: xxx(wait for edit)]{Introduction: xxx(wait for edit)} \label{sec:intro}

\begin{leftbar}
1..... (wait for edit)

2..... (wait for edit)
\end{leftbar}

3..... (wait for edit)

4..... (wait for edit)


%% -- Manuscript ---------------------------------------------------------------

%% - In principle "as usual" again.
%% - When using equations (e.g., {equation}, {eqnarray}, {align}, etc.
%%   avoid empty lines before and after the equation (which would signal a new
%%   paragraph.
%% - When describing longer chunks of code that are _not_ meant for execution
%%   (e.g., a function synopsis or list of arguments), the environment {Code}
%%   is recommended. Alternatively, a plain {verbatim} can also be used.
%%   (For executed code see the next section.)

\section{sec title (wait for edit)} \label{sec:models}

(wait for edit)

\begin{leftbar}
(wait for edit)
\end{leftbar}

\proglang{R} (wait for edit) \fct{glm} \citep{Chambers+Hastie:1992} in the
\pkg{stats} package. 
\begin{Code}
glm(formula, data, subset, na.action, weights, offset,
  family = gaussian, start = NULL, control = glm.control(...),
  model = TRUE, y = TRUE, x = FALSE, ...)
\end{Code}

(wait for edit)

\begin{leftbar}
(wait for edit)
\end{leftbar}

\begin{table}[t!]
\centering
\begin{tabular}{lllp{7.4cm}}
\hline
Type           & Distribution & Method   & Description \\ \hline
GLM            & Poisson      & ML       & Poisson regression: classical GLM,
                                           estimated by maximum likelihood (ML) \\
               &              & Quasi    & ``Quasi-Poisson regression'':
                                           same mean function, estimated by
                                           quasi-ML (QML) or equivalently
                                           generalized estimating equations (GEE),
                                           inference adjustment via estimated
                                           dispersion parameter \\
               &              & Adjusted & ``Adjusted Poisson regression'':
                                           same mean function, estimated by
                                           QML/GEE, inference adjustment via
                                           sandwich covariances\\
               & NB           & ML       & NB regression: extended GLM,
                                           estimated by ML including additional
                                           shape parameter \\ \hline
Zero-augmented & Poisson      & ML       & Zero-inflated Poisson (ZIP),
                                           hurdle Poisson \\
               & NB           & ML       & Zero-inflated NB (ZINB),
                                           hurdle NB \\ \hline
\end{tabular}
\caption{\label{tab:overview} Overview of various count regression models. The
table is usually placed at the top of the page (\texttt{[t!]}), centered
(\texttt{centering}), has a caption below the table, column headers and captions
are in sentence style, and if possible vertical lines should be avoided.}
\end{table}


%% -- Illustrations ------------------------------------------------------------

%% - Virtually all JSS manuscripts list source code along with the generated
%%   output. The style files provide dedicated environments for this.
%% - In R, the environments {Sinput} and {Soutput} - as produced by Sweave() or
%%   or knitr using the render_sweave() hook - are used (without the need to
%%   load Sweave.sty).
%% - Equivalently, {CodeInput} and {CodeOutput} can be used.
%% - The code input should use "the usual" command prompt in the respective
%%   software system.
%% - For R code, the prompt "R> " should be used with "+  " as the
%%   continuation prompt.
%% - Comments within the code chunks should be avoided - these should be made
%%   within the regular LaTeX text.

\section{third title (wait for edit)} \label{sec:illustrations}

(wait for edit)
%
\begin{Schunk}
\begin{Sinput}
R> data("quine", package = "MASS")
\end{Sinput}
\end{Schunk}
%
and a basic frequency distribution of the response variable is displayed in
Figure~\ref{fig:quine}.

\begin{leftbar}
(wait for edit)
\end{leftbar}

\begin{figure}[t!]
\centering
\includegraphics{ggESDA-visualization}
\caption{\label{fig:quine} Frequency distribution for number of days absent
from school.}
\end{figure}

(wait for edit)
%
\begin{Schunk}
\begin{Sinput}
R> m_pois <- glm(Days ~ (Eth + Sex + Age + Lrn)^2, data = quine,
+    family = poisson)
\end{Sinput}
\end{Schunk}
%
To account for potential overdispersion we also consider a negative binomial
GLM.
%
\begin{Schunk}
\begin{Sinput}
R> library("MASS")
R> m_nbin <- glm.nb(Days ~ (Eth + Sex + Age + Lrn)^2, data = quine)
\end{Sinput}
\end{Schunk}
%
In a comparison with the BIC the latter model is clearly preferred.
%
\begin{Schunk}
\begin{Sinput}
R> BIC(m_pois, m_nbin)
\end{Sinput}
\begin{Soutput}
       df      BIC
m_pois 18 2046.851
m_nbin 19 1157.235
\end{Soutput}
\end{Schunk}
%
Hence, the full summary of that model is shown below.
%
\begin{Schunk}
\begin{Sinput}
R> summary(m_nbin)
\end{Sinput}
\begin{Soutput}
Call:
glm.nb(formula = Days ~ (Eth + Sex + Age + Lrn)^2, data = quine, 
    init.theta = 1.60364105, link = log)

Deviance Residuals: 
    Min       1Q   Median       3Q      Max  
-3.0857  -0.8306  -0.2620   0.4282   2.0898  

Coefficients: (1 not defined because of singularities)
            Estimate Std. Error z value Pr(>|z|)    
(Intercept)  3.00155    0.33709   8.904  < 2e-16 ***
EthN        -0.24591    0.39135  -0.628  0.52977    
SexM        -0.77181    0.38021  -2.030  0.04236 *  
AgeF1       -0.02546    0.41615  -0.061  0.95121    
AgeF2       -0.54884    0.54393  -1.009  0.31296    
AgeF3       -0.25735    0.40558  -0.635  0.52574    
LrnSL        0.38919    0.48421   0.804  0.42153    
EthN:SexM    0.36240    0.29430   1.231  0.21818    
EthN:AgeF1  -0.70000    0.43646  -1.604  0.10876    
EthN:AgeF2  -1.23283    0.42962  -2.870  0.00411 ** 
EthN:AgeF3   0.04721    0.44883   0.105  0.91622    
EthN:LrnSL   0.06847    0.34040   0.201  0.84059    
SexM:AgeF1   0.02257    0.47360   0.048  0.96198    
SexM:AgeF2   1.55330    0.51325   3.026  0.00247 ** 
SexM:AgeF3   1.25227    0.45539   2.750  0.00596 ** 
SexM:LrnSL   0.07187    0.40805   0.176  0.86019    
AgeF1:LrnSL -0.43101    0.47948  -0.899  0.36870    
AgeF2:LrnSL  0.52074    0.48567   1.072  0.28363    
AgeF3:LrnSL       NA         NA      NA       NA    
---
Signif. codes:  0 '***' 0.001 '**' 0.01 '*' 0.05 '.' 0.1 ' ' 1

(Dispersion parameter for Negative Binomial(1.6036) family taken to be 1)

    Null deviance: 235.23  on 145  degrees of freedom
Residual deviance: 167.53  on 128  degrees of freedom
AIC: 1100.5

Number of Fisher Scoring iterations: 1


              Theta:  1.604 
          Std. Err.:  0.214 

 2 x log-likelihood:  -1062.546 
\end{Soutput}
\end{Schunk}



%% -- Summary/conclusions/discussion -------------------------------------------

\section{Summary and discussion} \label{sec:summary}

\begin{leftbar}
(wait for edit)
\end{leftbar}


%% -- Optional special unnumbered sections -------------------------------------

\section*{Computational details}

\begin{leftbar}
(wait for edit)
\end{leftbar}

(wait for edit)


\section*{Acknowledgments}

\begin{leftbar}
(wait for edit)
\end{leftbar}


%% -- Bibliography -------------------------------------------------------------
%% - References need to be provided in a .bib BibTeX database.
%% - All references should be made with \cite, \citet, \citep, \citealp etc.
%%   (and never hard-coded). See the FAQ for details.
%% - JSS-specific markup (\proglang, \pkg, \code) should be used in the .bib.
%% - Titles in the .bib should be in title case.
%% - DOIs should be included where available.

\bibliography{refs}


%% -- Appendix (if any) --------------------------------------------------------
%% - After the bibliography with page break.
%% - With proper section titles and _not_ just "Appendix".

\newpage

\begin{appendix}

(wait for edit)

\end{appendix}

%% -----------------------------------------------------------------------------


\end{document}
